% !TEX TS-program = pdflatex
% !TEX encoding = UTF-8 Unicode

\documentclass[11pt]{article} % use larger type; default would be 10pt

\usepackage[utf8]{inputenc} % set input encoding (not needed with XeLaTeX)
\usepackage[T1]{fontenc} %font encoding ? --> need to clarify that

%%% PAGE DIMENSIONS
\usepackage{geometry} % to change the page dimensions
\geometry{a4paper} % or letterpaper (US) or a5paper or....
\geometry{margin=1in} % for example, change the margins to 2 inches all round

\usepackage{graphicx} % support the \includegraphics command and options
\usepackage[parfill]{parskip} % Activate to begin paragraphs with an empty line rather than an indent

%%% PACKAGES
\usepackage{booktabs} % for much better looking tables
\usepackage{array} % for better arrays (eg matrices) in maths
\usepackage{paralist} % very flexible & customisable lists (eg. enumerate/itemize, etc.)
\usepackage{verbatim} % adds environment for commenting out blocks of text & for better verbatim
\usepackage{subfig} % make it possible to include more than one captioned figure/table in a single float
% These packages are all incorporated in the memoir class to one degree or another...

%%% HEADERS & FOOTERS
\usepackage{fancyhdr} % This should be set AFTER setting up the page geometry
\pagestyle{fancy} % options: empty , plain , fancy
\renewcommand{\headrulewidth}{0pt} % customise the layout...
\lhead{}\chead{}\rhead{}
\lfoot{}\cfoot{\thepage}\rfoot{}
%%% END Article customizations

%%% The "real" document content comes below...

\title{PSZT - przeszukiwanie}
\author{Stawczyk Przemysław 293153, Piotr Zmyślony 268833}
\date{} % Activate to display a given date or no date (if empty),
         % otherwise the current date is printed 

\begin{document}
\maketitle

\section{Opis Zagadnienia}
\subsection{Treść zadania}
\paragraph{}
Zaimplementować i przetestować algorytm \textsl{A*} dla zadania znalezienia ścieżki o najmniejszej wadze od punktu A do B. Wejściem aplikacji jest plik z listą krawędzi grafu (dla każdej krawędzi zdefiniowany jest punkt początkowy, końcowy I waga krawędzi). Wyjściem aplikacji jest najkrótsza ścieżka od punktu A do punktu B. Porównać działanie algorytmu \textsl{A*}  z \textsl{brutalnym} przeszukiwaniem grafu. Zastosowanie dodatkowego algorytmu będzie dodatkowym atutem przy ocenie projektu.
\subsection{Narzędzia}
Skrypty oraz algorytm zostały zaimplementowane w Pythonie 3. \\
Wykorzystano biblioteki: \textsl{xml, math, time, json, requests, heapq }
\subsection{Realizacja}
Po konsultacji wybrane dane wejściowe to grafy  \textsl{Polska} oraz  \textsl{Germany50} ze strony sndlib.zib.de. Dane przedstawione w formacie XML zostają przeprocesowane na plik XML, sformatowany w nieco inny sposób: wierzchołki (miasta) kodowane identycznie, krawędzie nie zawierają id, pojemności i kosztu, a odległość pomiędzy miastami, pominięte zostają również  \textsl{zapotrzebowania} pomiędzy poszczególnymi miastami. 
Dane o odległości drogowej pobierane są z wykorzystaniem API serwisu \textsl{here.com}, a w przypadku błędu w odpowiedzi API, odległość wyliczana jest formułą haversine, która słóży również jako heurystyka dla algorytmu \textsl{A*}.
\subsection{Heurystyka dla A*}
Heurystyka używana przez algorytm \textsl{A*}  musi pozwalać na oszacowanie prawdopodobieństwa wyniku pozostałej części procesu analizy składniowej, przy ustalonym już, uprzednio przeanalizowanym fragmencie. Musi być problemem prostszym niż problem ogólny \textsl{[nie może być to np. rozwiązanie podproblemu algorytmem wymagajacym czasowo w stosunku do pierwotnego]}. Wynik działania funkcji powinien stanowić ograniczenie dolne kosztu dotarcia do celu \textsl{[nie da się tam dotrzeć szybciej, ale może być tak, że da się wolniej]}. W szczególnym przypadku, gdy heurystyka zwraca wartość 0 algorytm sprowadza się do algorytmu Dijkstry.\\
Wymagania spełnia formuła haversine zastosowana w naszej implementacji tego algorytmu.
\paragraph{Formuła haversine}
Formuła ta pozwala na otrzymanie przybliżonej długości łuku, który łączy dwa punktu na kuli, używając jako dane wejściowe szerokości i długości geograficznych obu punktów.
\begin{center}
\[ d = 2r \arcsin\left(\sqrt{\sin^2\left(\frac{\varphi_2 - \varphi_1}{2}\right) + \cos(\varphi_1) \cos(\varphi_2)\sin^2\left(\frac{\lambda_2 - \lambda_1}{2}\right)}\right) \]
\end{center}
W powyższym wzorze \textsl{r} to promień Ziemi, $\varphi$ i $\lambda$ to odpowiednio szerokości i długości geograficzne.
\section{Implementacja}
\subsection{Program}
Skrypt zawierający algorytmy przyjmuje 4 argumenty : \\ 
\textsl{<graf> <tryb> <początek> <cel>}\\
przy czym tryby pracy programu są następujące:
\begin{enumerate}
\item
alg. brutalny (przeszukiwanie wszystkich możliwych ścieżek)
\item
algorytm A* 
\item
algorytm Dijkstry
\item
A* oraz algorytm Dijkstry
\item
Wszystkie 3 algorytmy
\item
Tryb testowy, wymaga jedynie podania argumentów w postaci \textsl{<graf> 6 <ilość prób testowych>}. Porówuje wydajności A* oraz algorytmu Dijkstry na podstawie określonej ilości danych testowych (generowanych automatycznie).
\end{enumerate}
Wynikiem działania dla wszystkich opcji są ścieżki, jej koszt (odległość) oraz czas działania. Czas działania jest mierzony tylko w momencie działania algorytmu. Funkcja heurystyczna korzysta z formuły haversine obliczającą odległość w linii prostej po powierzchni sfery pomiędzy miastami. Funkcja ta spełnia wymagania heurystyki w algorytmie A* gdyż stanowi dolne ograniczenie odległości pomiędzy miastami \textsl{[nie da się dotrzeć szybciej niż w linii prostej]}
\subsection{Testy działania}
Poniższa tabelka przedstawia czasy działania algorytmu dla jednej procesu wyszukiwania jednej ścieżki, uśrednione na podstawie 10000 powtórzeń dla A* i algorytmu Dijkstry.
\begin{table}[h!]
\centering
\begin{tabular}{ ||p{3cm}||p{5cm}|p{5cm}||  }
\hline
\multicolumn{3}{|c|}{Uśrednione wyniki} \\
\hline
Algorytm & średni czas dla grafu Polska & średni czas dla Germany50 \\
\hline
Brutalny &375 $\mu s$ &>15 minut \\
Dijkstra &19 $\mu s$ &87 $\mu s$ \\
A* &25 $\mu s$ &41 $\mu s$ \\
\hline
\end{tabular}
\end{table}
\subsection{Analiza wyników}
Algorytm Dijkstry oraz \textsl{A*} są zdecydowanie szybsze od podejścia brutalnego, którego użycie dla grafów o większej ilości krawędzi i wierzchołków może prowadzić do złożoności \textsl{O(n!)}. Dodatkowo dzięki zastosowaniu algorytmu heurystycznego algorytm \textsl{A*} sprawdza jedynie potencjalnie najlepsze ścieżki co ogranicza \textsl{rozprzestrzenianie} się algorytmu i zdecydowanie przyspiesza jego pracę dla odpowiednio dużych grafów. Jako, że funkcja haversine jest umiarkowanie skomplikowaną formułą to dla mapy Polski z 12 miastami i 18 krawędziami wzrost wydajności jest znikomy, ale już dla bardziej realistycznego zastosowania, w grafie dla mapy Niemiec, o 50 wierzchołkach znajduje dłuższe trasy szybciej.


\end{document}