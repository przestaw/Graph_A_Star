% !TEX TS-program = pdflatex
% !TEX encoding = UTF-8 Unicode

\documentclass[11pt]{article} % use larger type; default would be 10pt

\usepackage[utf8]{inputenc} % set input encoding (not needed with XeLaTeX)
\usepackage[T1]{fontenc} %font encoding ? --> need to clarify that

%%% PAGE DIMENSIONS
\usepackage{geometry} % to change the page dimensions
\geometry{a4paper} % or letterpaper (US) or a5paper or....
\geometry{margin=1in} % for example, change the margins to 2 inches all round

\usepackage{graphicx} % support the \includegraphics command and options
\usepackage[parfill]{parskip} % Activate to begin paragraphs with an empty line rather than an indent

%%% PACKAGES
\usepackage{booktabs} % for much better looking tables
\usepackage{array} % for better arrays (eg matrices) in maths
\usepackage{paralist} % very flexible & customisable lists (eg. enumerate/itemize, etc.)
\usepackage{verbatim} % adds environment for commenting out blocks of text & for better verbatim
\usepackage{subfig} % make it possible to include more than one captioned figure/table in a single float
% These packages are all incorporated in the memoir class to one degree or another...

%%% HEADERS & FOOTERS
\usepackage{fancyhdr} % This should be set AFTER setting up the page geometry
\pagestyle{fancy} % options: empty , plain , fancy
\renewcommand{\headrulewidth}{0pt} % customise the layout...
\lhead{}\chead{}\rhead{}
\lfoot{}\cfoot{\thepage}\rfoot{}
%%% END Article customizations

%%% The "real" document content comes below...

\title{PSZT - przeszukiwanie}
\author{Stawczyk Przemysław 293153, Piotr Zmyślony}
\date{} % Activate to display a given date or no date (if empty),
         % otherwise the current date is printed 

\begin{document}
\maketitle

\section{Opis Zagadnienia}
\subsection{Treść zadania}
\paragraph{}
Zaimplementować i przetestować algorytm \textsl{A*} dla zadania znalezienia ścieżki o najmniejszej wadze od punktu A do B. Wejściem aplikacji jest plik z listą krawędzi grafu (dla każdej krawędzi zdefiniowany jest punkt początkowy, końcowy I waga krawędzi). Wyjściem aplikacji jest najkrótsza ścieżka od punktu A do punktu B. Porównać działanie algorytmu \textsl{A*}  z \textsl{brutalnym} przeszukiwaniem grafu. Zastosowanie dodatkowego algorytmu będzie dodatkowym atutem przy ocenie projektu.
\subsection{Narzędzia}
Skrypty oraz algorytm zostały zaimplementowane w Pythonie 3. \\
Wykorzystano biblioteki: \textsl{xml, math, time, json, requests, heapq }
\subsection{Realizacja}
Po konsultacji wybrane dane wejściowe to grafy  \textsl{Polska} oraz  \textsl{Germany50} ze strony sndlib.zib.de. Dane przedstawione w formie xml zostają przeprocesowane na plik  \textsl{xml} sformatowany w nieco inny sposób: wierzchołki (miasta) kodowane identycznie, krawędzie nie zawierają id, pojemności i kosztu, a odległość pomiędzy miastami, pominięte zostają również  \textsl{zapotrzebowania} pomiędzy poszczególnymi miastami. 
Dane o odległości pobierane są z wykorzystaniem API webowego, w przypadku błędu w API odległość wyliczana jest metodą haversine która słóży również jako heurystyka dla algorytmu A*
\subsection{Heurystyka dla A*}
Heurystyka używana przez algorytm \textsl{A*}  musi pozwalać na oszacowanie prawdopodobieństwa wyniku pozostałej części procesu analizy składniowej, przy ustalonym już, uprzednio przeanalizowanym fragmencie. Musi być problemem prostszym niż problem ogólny \textsl{[nie może być to np rozwiązanie podproblemu algorytmem wymagajacym czasowo w stosunku do pierwotnego]}. Wynik działania funkcji powinien stanowić ograniczenie dolne kosztu dotarcia do celu \textsl{[nie da się tam dotrzeć szybciej, ale może być tak, że da się wolniej]}. W szczególnym przypadku, gdy heurystyka zwraca wartość 0 algorytm sprowadza się do algorytmu Dijkstry.\\
Wymagania spełnia formuła haversine zastosowana w naszej implementacji tego algorytmu.
\paragraph{Haversine}

\section{Implementacja}
\subsection{Program}
Skrypt zawierający algorytmy przyjmuje 4 argumenty : \\ 
\textsl{<graf> <początek> <cel> <tryb>}\\
przy czym tryb jest opcjonalny i może obejmować:
\begin{enumerate}
\item
alg. brutalny
\item
A* \textsl{domyślny}
\item
Dijkstra
\item
A* oraz Dijkstra
\item
Wszystkie 3 algorytmy
\end{enumerate}
A jego wynikiem jest podana ścieżka, koszt ścieżki oraz czas działania. Czas działania jest mierzony tylko w momencie działania algorytmu. Funkcja heurystyczna korzysta z formuły haversine obliczającą odległość w linii prostej po powierzchni sfery pomiędzy miastami. Funkcja ta spełnia wymagania heurystyki w algorytmie A* gdyż stanowi dolne ograniczenie odległości pomiędzy miastami \textsl{[nie da się dotrzeć szybciej niż w linii prostej]}
\subsection{Testy działania}
\begin{table}[h!]
\centering
\begin{tabular}{ ||p{3cm}||p{5cm}|p{5cm}||  }
\hline
\multicolumn{3}{|c|}{Uśrednione wyniki} \\
\hline
Algorytm & średni czas dla grafu Polska & średni czas dla grafu Germany50 \\
\hline
Brutalny & 0 &0 \\
Dijkstra & 0 &0 \\
A* & 0 &0 \\
\hline
\end{tabular}
\end{table}
\subsection{Analiza wyników}
Algorytm Dijkstry oraz \textsl{A*} są zdecydowanie szybsze od podejścia brutalnego. Dodatkowo dzięki zastosowaniu algorytmu heurystycznego algorytm \textsl{A*} sprawdza jedynie potencjalnie najlepsze ścieżki co ogranicza \textsl{rozprzestrzenianie} się algorytmu i zdecydowanie przyspiesza jego pracę dla odpowiednio dużych grafów. Jako, że funkcja haversine jest umiarkowanie skomplikowaną formułą to dla mapy polski z 12 miastami i 18 krawędziami daje gorsze rezultaty niż sprawdzenie wszystkich ścieżek o długości mniejszej od optymalnej, ale już dla bardziej realistycznego zastosowania w grafie niemiec o 50 wierzchołkach znajduje dłuższe trasy szybciej.


\end{document}